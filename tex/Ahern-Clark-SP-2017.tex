\documentclass[linguex]{sp}
% possible options: [cm] for Computer Modern font instead of Times
%                   [linguex] to load the linguex example package
% e.g.: \documentclass[cm,linguex]{sp}


\usepackage{amssymb,amsthm,amsfonts}
\theoremstyle{definition} \newtheorem{definition}{Definition} 
\usepackage{sectsty}
\usepackage{fancyhdr}
\usepackage{multirow}
\usepackage{pstricks}
\usepackage{graphicx}

\usepackage{booktabs}
\usepackage{longtable}
\usepackage{array}


\usepackage{subfigure}
\usepackage{tikz}
\usetikzlibrary{arrows,automata,chains,matrix,positioning,scopes}
\usepackage[english,greek]{betababel}

\usepackage{natbib}

\graphicspath{{../local/out/}}

% \pdf* commands provide metadata for the PDF output. ASCII characters only!
%\pdfauthor{Christopher Ahern}
\pdfauthor{}
%\pdftitle{Ahern-Clark-SP-2015}
\pdftitle{}
\pdfkeywords{Language change, Negation, Jespersen's cycle, Evolutionary game theory}

% Optional short title inside square brackets, for the running headers.
% If no short title is given, no title appears in the headers.
\title[Conflict, cheap talk, and Jespersen's cycle]{Conflict, cheap talk, and Jespersen's cycle\thanks{We would like to thank Michael Franke and two anonymous S\&P reviewers for comments and suggestions that have substantially improved the paper. We would also like to thank Mitchell Newberry and Josh Plotkin for helpful discussions and advice. We gratefully acknowledge funding from the University of Pennsylvania that made this work possible.}}

% Optional short author inside square brackets, for the running headers.
% If no short author is given, no authors print in the headers.
\author[Ahern \& Clark]{ 
  % As many authors as you like, each separated by \AND.
  \spauthor{Christopher Ahern \\ \institute{Department of Linguistics,\\ University of Pennsylvania}} \AND
  \spauthor{Robin Clark \\ \institute{Department of Linguistics,\\ University of Pennsylvania}}
}


\begin{document}

\maketitle

\begin{abstract}
Game-theory has found broad application in modeling meaning in both the classical Gricean case of common interests between interlocutors and, more recently, in cases of conflicting interests. Here we consider how conflicting interests between speakers and hearers can be used to explain language change. We use tools from  evolutionary game theory to characterize the effect of conflicting interests in the case of Jespersen's cycle.  We show how the cycle can be modeled as an inflationary process due to signaling with costless signals under conflicting interests. We fit the resulting dynamic model to time series data drawn from a historical corpus of Middle English.
\end{abstract}

\begin{keywords}
	 Language change, Historical Linguistics, Jespersen's cycle, Evolutionary game theory
\end{keywords}

%\noindent
%Word count: 10,258

\section{Introduction}
\label{Introduction}


This paper presents a dynamic evolutionary game-theoretic model of a population that links conflicting interests between speakers and hearers to one aspect of the historical change in sentential negation referred to as \emph{Jespersen's cycle} \citeyearpar{jespersen:1917}: an initially emphatic form of negation increases in frequency over time and  loses its emphasis. We model the change in the meaning of negation as a kind of \emph{inflationary} process in which the conditions of use for a particular linguistic signal are extended over time \citep{dahl:2001}. We argue that this increase in frequency and loss of information can be explained by the fact that language is used both to convey information and negotiate social status \citep{dessalles2007,franke-etal:2012}. Simply put, hearers benefit when speakers share useful information, but  speakers also benefit socially when hearers confer status by listening to what they have to say. Under these conditions, the interests of speakers and hearers are largely, but not perfectly aligned (cf. \citealt{benz-jager-van-rooij:2006, franke-etal:2012, de-jaegher-van-rooij:2014, asher2013strategic}). Using tools from \emph{evolutionary game theory} \citep{maynard-smith1982, hofbauer-sigmund1998}, we model how the interaction between speakers and hearers changes the information carried by signals over time, and fit the resulting dynamic model to corpus data from Middle English.

In recent work, different kinds of dynamic models have been applied to particular instances of historical change \citep{schaden2012, deo2015, yanovich2015, enke2016}. The most closely related to our work is \cite{schaden2012}, which models the change in the present perfect as an inflationary process that arises from speakers overestimating the relevance of their conversational contributions. Here we take the change in the meaning of sentential negation to be an inflationary process driven by the interests of speakers and hearers not being perfectly aligned. There are two crucial contributions of this work. First, we offer a mechanism for the observed increase in the frequency of different forms of negation, which has often been assumed but not explicitly formulated (cf. \citealt{detges-waltereit2002, hopper-traugottt2003, kiparsky-condoravdi:2006}, \emph{inter alia}).  Second, fitting the dynamic model to a time series drawn from historical corpus data allows us to consider more than just the parameter values that generate the qualitative patterns of change. Namely, it yields quantitative estimates of model parameters, which can be further compared to corpus and experimental data.  These kinds of insights will be an important contribution to historical linguistics where the use of quantitative methods is well-established, but explicit mechanisms for change are absent \citep[4]{kroch1989}. Importantly, positing mechanisms that underly language change and estimating the relevant parameters of dynamic models allows us to compare models in terms of fit, complexity, and interpretability \citep{burnham2003}. 

The rest of the paper is organized as follows. In Section \ref{Jespersen's cycle}, we present a general overview of Jespersen's cycle, noting that the term is used to refer to two often-related but distinct processes: the \emph{formal cycle} describes the forms of negation over time; the \emph{functional cycle} describes how those forms are used to signal information over time. We note the logical relationship between the two cycles and focus on functional cycle as an inflationary process. In Section \ref{Signaling} we model the interaction between speakers and hearers, as well as their respective interests, as a \emph{signaling game} with costless signals, or \emph{cheap talk}. In Section \ref{Equilibrium} we determine the \emph{evolutionarily stable strategies} of the signaling game, which correspond to speaker and hearer behaviors in a population that are resistant to change. In Section \ref{Dynamics} we consider signaling in a population over time under a particular \emph{evolutionary game dynamics} with conflicting interests . We fit the resulting model to corpus data from the functional cycle in Middle English and discuss the fitted parameters. In Section \ref{Conclusion} we conclude by discussing the implications of the model and directions for future research.


\section{Jespersen's cycle}
\label{Jespersen's cycle}


Originally coined by \citet[88]{dahl:1979}, the term \emph{Jespersen's cycle} is often used in reference to the following observation regarding sentential negation in several languages made by \citet[4]{jespersen:1917}.\footnote{Sentential negation refers to the semantic property of negating an entire proposition, not just some subpart. It can be distinguished from morphological (e.g., \emph{un-}, \emph{non-}) and constituent negation (e.g., \emph{John might have not understood}) using diagnostics such as tag questions and performative paraphrases. Sentential negation is almost always syntactically expressed as a negative phrase and can be distinguished from negative quantifiers (e.g., \emph{nothing}) using \emph{wh}-substitution.}

\begin{quote}
The history of negative expressions in various languages makes us witness the following curious fluctuation: the original negative adverb is first weakened, then found insufficient and therefore strengthened, generally through some additional word, and this in its turn may be felt as the negative proper and may then in course of time be subject to the same developments as the original word.
\end{quote}
However, this passage can and has been interpreted in two very distinct ways \citep{vanderAuwera2009}. The fundamental ambiguity stems from the fact that Jespersen noted both \emph{formal} and \emph{functional} patterns in the expression of sentential negation over time. Both patterns can be conceived of as cycles in their own right. That is, we can define a series of transitions from and back to states that are in some sense formally or functionally equivalent. 

But, the term \emph{Jespersen's cycle} is often used to refer to one or the other of these aspects, or both of them simultaneously. For example, the canonical presentation of what is referred to as Jespersen's cycle conflates these two uses (cf. \citealt{posner1985,schwegler1988,ladusaw1993}). The following stages are used to characterize the cycle:

\begin{itemize}
    \item [1.] \textsc{\textcolor{red}{neg V}}
    \item [2.]  \textsc{\textcolor{red}{neg V} \textcolor{blue}{(neg)}}
    \item [3.] \textsc{\color{blue} neg V neg}
    \item [4.] \textsc{\color{green} V neg}
\end{itemize}
At the start of the cycle, negation is expressed by a single pre-verbal element. At the second stage of the cycle an optional post-verbal element, indicated by parentheses, which creates an emphatic bipartite form.\footnote{Note that the optional post-verbal element need not be a negative marker \emph{per se}. Various elements, including negative polarity items that are not negative markers themselves, can be added to create bipartite forms (cf. \citealt{horn:1989,givon1978, croft1991}).} At the third stage of the cycle the post-verbal element is obligatory and the bipartite form ceases to be emphatic. At the fourth stage we see the loss of the pre-verbal element resulting in a post-verbal form.

In this section we define and distinguish the two uses of the term that are intertwined in this representation, which we will call the \emph{formal} and \emph{functional} Jespersen cycles. In short, the distinction is between changes in the forms of negation available and changes in how those forms are used to signal information, respectively.  First, we outline the formal aspects of how negation is expressed at the stages of the formal cycle. Second, we outline the function of those forms at different stages of the functional cycle. Finally, we note that the functional cycle falls into a broader class of inflationary processes observed in language change.

The formal cycle is defined in terms of the forms that are used to express negation over time, and consists of two transitions. We see the first stage of the formal cycle in the history of English with the pre-verbal \emph{\textcolor{red}{ne}} in Old English, which expresses sentential negation alone.

\exg. Ic \textcolor{red}{ne} secge\\
      I \textsc{neg} say\\
      (Old English)

This is followed by a transition to the bipartite form where another negative element is added. This is seen in Middle English, where \emph{\textcolor{blue}{ne}} is supplemented by \emph{\textcolor{blue}{not}}.

\exg. I \textcolor{blue}{ne} seye \textcolor{blue}{not}\\
      I \textsc{neg} say \textsc{neg}\\
      (Middle English)

The final stage in the formal cycle is a return to a single negative element. In fact, all three forms are present and overlap in Middle English. But, in Early Modern English the preverbal element in \emph{\textcolor{blue}{ne...not}} is lost, leaving the post-verbal form \emph{\textcolor{green}{not}}. 

\exg. I say \textcolor{green}{not}\\
      I say \textsc{neg}\\
      (Early Modern English)

So, the formal cycle consists of the increase and then decrease in the formal complexity of sentential negation. It is cyclic in the sense that the forms of negation at the start and end are of equal formal complexity. Before moving on, there are two important points to emphasize.

First, the emergence of \emph{do}-support in Early Modern English yields a state parallel to Old English with a sole pre-verbal negator, but it is not a necessary component of the formal cycle.\footnote{\citet[10]{jespersen:1917} noted the uniqueness of these developments, which he attributed to a tendency to place negation at the beginning of the sentence to avoid confusion on the part of hearers. Yet, despite this purported tendency, the majority of languages that Jespersen noted, including his native Danish, persist in a supposedly perplexing state of purely post-verbal negation.} That is, while negation in Present-day English may be structurally parallel to Old English, negation has been formally parallel since the loss of the bipartite form \emph{\textcolor{blue}{ne...not}}.

\exg. I do\textcolor{red}{n't} say\\
      I do-\textsc{neg} say\\
      (Present-day English)

As a point of comparison, we observe the same transition from pre-verbal \emph{\textcolor{red}{ne}} to bipartite \emph{\textcolor{blue}{ne...pas}} to post-verbal \emph{\textcolor{green}{pas}} forms of negation in French, but no subsequent transition back to pre-verbal negation. When it comes to the formal cycle it matters how much material is used to express negation, not necessarily where that material stands in relation to the verb.

Second, the formal cycle is not necessarily the transition from pre- to post-verbal negation. That is, the bipartite form is not necessarily formed through the addition of a post-verbal element. For example, in modern African American Vernacular English, negation can be supplemented by a pre-verbal element \emph{eem}, which can also express negation in its own right \citep{jones2016, jones2015toward}.

\ex. You do\textcolor{blue}{n't eem} know.

\ex. You \textcolor{red}{eem} know.

The formal cycle could just as well be from post- to pre-verbal negation. It is a contingent historical fact, arising from the syntax of  English and the source of the additional material, rather than some necessary property of the formal cycle. Again, it is the formal status rather than position relative to the verb that is relevant.

While the formal cycle is defined by the forms of negation over time, the functional cycle is defined by how those forms are put to use, and is characterized by two transitions.  The first transition occurs with the introduction of a stronger more emphatic negative form, often the result of adding a \emph{negative polarity item} or other material to the original plain form \citep{horn:1989, givon1978, croft1991}. The initial effect of the bipartite form, in Jespersen's words \citeyearpar[15]{jespersen:1917}:\footnote{Despite the evocative phrasing, Jespersen was certainly not the first to notice the trajectory of the functional cycle. \cite{vanderAuwera2009} suggests \emph{Meillet's spiral} \citeyearpar[394]{meillet1912} as a potentially more appropriate term for the functional cycle, to which we might add \emph{Gardiner's gyre} \citeyearpar[134]{gardiner1904}.}

\begin{quotation}
...[I]n most cases the addition serves to make the negative more impressive as being more vivid or picturesque, generally through an exaggeration, as when substantives meaning something very small are used as subjuncts.
\end{quotation}

The second transition of the functional cycle occurs as the new emphatic form increases in frequency relative to the incumbent form, weakens in intensity, and replaces the original negative form. The functional cycle consists of one form of plain negation is replaced by another form. It is cyclic in the sense that the number of functionally distinct forms of negation increases then decreases. For example, in the history of English and French an initially emphatic bipartite form weakens over time and comes to have the same force as the pre-verbal form.

There are two important points to be made with regard to the functional cycle and its relationship to the formal cycle. First, the conflation of the formal and functional cycles understandably stems from the fact that the functional cycle often coincides with the first transition of the formal cycle.  Intuitively, \emph{\textcolor{blue}{ne...not}} is a more complex form than  \emph{\textcolor{red}{ne}}, and thus we would expect it have a more restricted and hence stronger meaning. Note that this does not apply to the second transition of the formal cycle given that the same relationship between \emph{\textcolor{green}{not}} and \emph{\textcolor{blue}{ne...not}} does not hold.  Second, while the functional cycle often takes place within the first transition of the formal cycle, it can occur entirely independently of the formal cycle.   For instance, one form can be replaced by another of equal formal complexity.  Or, in Meillet's \citeyearpar[134]{meillet1912} estimation, the functional cycle is achieved when ``one adds  new \emph{or} different words''.  

\cite{kiparsky-condoravdi:2006} note that this is exactly what takes place in the history of Greek. Historical forms of negation in Greek are listed in Table \ref{greek-table}, where emphatic negation is taken to be the form with a more restricted meaning in comparison to \emph{plain} negation at any point in time.\footnote{We omit some of the forms for a concise presentation (cf. \citealt[1]{kiparsky-condoravdi:2006}). We return to the point of defining what we mean by \emph{emphatic negation} and \emph{emphasis} in Section \ref{Signaling}. However, for interpreting the table it is only crucial that a single form displaces the old plain form.} The sources of the different forms are ordered chronologically by row. There is a consistent transition of forms between the two functions: the emphatic negation of the last century or millennium becomes the plain negation of this century or millennium. Crucially, at least some of these functional cycles occur without any concomitant formal cycle.  For example, if we compare the formal complexity from Early Medieval Greek onwards, they would all be equivalent. All of them consist of a shared pre-verbal element \textgreek{d'en} along with a single post-verbal element. Thus, we see several bipartite forms come to express plain negation over time.

\begin{table}
\begin{center}
\begin{tabular}{ccc}
\toprule
\cmidrule(r){1-2}
\textsc{plain} & \textsc{emphatic} & \textsc{source} \\
\midrule
      \textgreek{o\~>u...ti} & \textgreek{o\~>u-de...en} & Ancient Greek \\
      \textgreek{(o>u)d'en...ti} & \textgreek{d'en...t'ipote} & Early Medieval Greek \\
      \textgreek{d'en...t'ipote} & \textgreek{d'en... pr\~ama} & Greek Dialects \\
      \textgreek{d'en...pr\~ama} & \textgreek{den...>apantoq'h} & Modern Cretan \\
\bottomrule
\end{tabular}
\end{center}
    \caption{Historical forms of plain and emphatic negation in Greek}
    \label{greek-table}

\end{table}

So, the formal and functional cycles are often closely intertwined, but they are in fact distinct phenomena. This means that the facts to be explained for each are also distinct.  We focus on the functional cycle rather than the formal cycle, which in the case of English, means that our goal will be to provide a model that explains the transition from \emph{\textcolor{red}{ne}} to \emph{\textcolor{blue}{ne...not}}. Now, given the history of English, this change is closely tied up with the subsequent transition in the formal cycle from \emph{\textcolor{blue}{ne...not}}  to  \emph{\textcolor{green}{not}}. This is an important point, which we return to in Section \ref{Dynamics}. However, it should be emphasized that our goal is not to model of the formal cycle, which consists of the transitions both from \emph{\textcolor{red}{ne}} to \emph{\textcolor{blue}{ne...not}} and from \emph{\textcolor{blue}{ne...not}}  to  \emph{\textcolor{green}{not}}. This point bears repeating insofar as the two cycles are so often conflated.

In what follows  we treat the functional cycle as a kind inflationary processes where its context for use is expanded over time along some evaluative scale \citep{dahl:2001}. The bipartite form in English may initially be emphatic, but as it becomes obligatory it ceases to be. For any form, if it is the only one in use, then it cannot carry any special meaning. There is nothing else to be special in comparison to. As \citet[5]{kiparsky-condoravdi:2006} rightly put it, ``to emphasize everything is to emphasize nothing.''  While we can conceive of the functional cycle as an inflationary process, this still leaves us with the question of \emph{why} the inflation happens. In the next section we make clear what we mean by \emph{emphatic negation}, what using emphatic negation signals about speakers, and what social consequences that information might have.


\section{A signaling game model of emphasis}
\label{Signaling}

The notion of emphasis is central to understanding the functional cycle as an inflationary process. In the context of emphatic negation, emphasis has generally been defined in terms of informativity (cf. \citealt{krifka1995polarity, detges-waltereit2002, kiparsky-condoravdi:2006}, \emph{inter alia}).  Indeed, we define \emph{emphatic negation} as a form of sentential negation that is informative in a particular sense. In what follows we show that the definition of emphasis as informativity arises naturally when we model the interaction between speakers and hearers as a \emph{signaling game} \citep{lewis:1969}. In this section we introduce the abstract structure of signaling games before turning to defining the various components of the game in terms relevant to emphatic negation and the functional cycle. 

In the abstract, a signaling game is played between a sender and a receiver, which in this case correspond to a speaker and a hearer. It has the following basic structure. First, the sender has some private information, $t \in T$, drawn according to a prior distribution, $p(t)$. This information is referred to as the sender's state and can generally be thought of as some form of information about the world. Next, the sender chooses a message, $m \in M$, to send to the receiver according to a strategy that maps states to messages, $s \co T \rightarrow M$. Finally, the receiver takes some action, $a \in A$, in response to the message according to a strategy that maps messages to actions, $r \co M \rightarrow A$. Both senders and receivers have preferences over combinations of states and actions, which are represented by the utility functions that map the combination of states and actions to real numbers, $U_S \co T \times A \rightarrow \mathbb{R}$ and  $U_R \co T \times A \rightarrow \mathbb{R}$, respectively. Senders and receivers prefer outcomes that yield higher utilities according to these functions.

Turning to emphatic negation, we start by defining the states of the game, which characterize the private information of the speaker. In particular, we want to capture the intuition that forms of emphatic negation are highly informative. Because negative polarity items are so often used in new forms of emphatic negation, we take the analysis of polarity items in \cite{krifka1995polarity} and  \cite{eckardt2006} as a starting point. Namely, the use of a negative polarity item induces  a scale of ordered alternatives \citep{rooth1992}, and picks out the endpoint of that scale, which entails all other alternatives on the scale \citep{fauconnier1975, horn:1989}. For example, both \emph{generalizers} (\emph{at all}) and \emph{minimizers} (\emph{a crumb}) create scales of alternatives in the following sentences.

\ex. \label{all}	\a. John didn't eat \emph{at all}. \label{gen}
	\b. John didn't eat \emph{a crumb}. \label{min}
	\b. John didn't eat \label{plain}

In \ref{gen} the scale consists of more or less strict ways of interpreting \emph{eat}. In this case, \emph{at all} signals that there was no \emph{eating} even under the least strict interpretation, which sits at one end of the scale. If \emph{eating} does not hold under the least strict interpretation, then it does not hold for any stricter interpretations. In \ref{min} the scale consists of quantities of food. In this case, not eating \emph{a crumb} entails not eating \emph{a bite}, \emph{a snack}, and \emph{a meal}. Given that the use of both \ref{gen} and \ref{min} entails the use of all other alternatives on their respective scales, both forms are more informative than plain negation in \ref{plain}.

However, the scales for these different kinds of polarity items differ, so we cannot simply map them to a shared set of states. Rather, we want to find a shared scale that underlies both and captures the conceptual similarity between the two. We can define such a shared scale by making the following assumptions. First, we assume that there are objective facts about the world that the speaker observes. For example, in the case of \ref{all}, we might consider scenarios where the speaker saw John briefly in the afternoon, ate lunch with John, or spent the whole day with John. Second, we assume that speakers form a subjective estimate of the truth of a proposition given these facts: speakers form an estimate of the \emph{standard of evidence} they have for asserting a given proposition (cf. \citealt{lewis1970,krifka1995polarity}). Third, we assume that given the same set of observations about the world that different speakers will form the same subjective estimate of the standard of evidence. That is,  the standard of evidence is what a \emph{reasonable} person would arrive at given the observed facts about the world. This means that the speaker's standard of evidence is subjective, but overwhelmingly determined by objective facts about the world. Finally, we assume that speakers abide by Grice's \citeyearpar{grice1975} \emph{maxim of quality}, only making assertions for which they have evidence.  In what follows we take the speaker's private information to be a standard of evidence drawn from the set of states, $T : [0,1]$, where the state  $t_0$ is some minimum standard of evidence required to truthfully assert a proposition and the state  $t_1$ is the strictest standard of evidence possible.\footnote{These assumptions obviously abstract from the interesting possibilities of deception or differences in reasoning across individual contexts and speakers, or the possibility of bias in speakers reasoning about the relevant evidence. To address these possibilities we could, for example, consider a state space that includes standards of evidence for and against a proposition to allow for deception, use a more complicated state space that incorporates contextual information or model different contexts with different priors, or include a bias in the speakers estimation of the standard of evidence that deviates from the prior  (cf. \citealt{schaden2012}).} That is, the states range from a shred of evidence for asserting a proposition to definitive proof beyond the shadow of a doubt.

These assumptions allow for a uniform treatment of emphasis with different kinds of negative polarity items. For example, reconsider the scenarios mentioned above of a speaker having seen John briefly in the afternoon, having met for lunch, or having spent the entire day together. Now, suppose that the speaker did not observe John eating in any of these scenarios.  At the end of the day we can order these scenarios in terms of the standard of evidence that a reasonable speaker would estimate for the claims in \ref{all}. Seeing John briefly in the afternoon leaves the rest of the day unaccounted for, eating lunch with John leaves most of the day including other meals unaccounted for, but spending the entire day with John does not not leave much if any time unaccounted for. It is obvious that briefly seeing John not eating would at best constitute extremely weak evidence for any of the claims in \ref{all}, seeing John not eating at lunch would constitute slightly stronger evidence, and spending the entire day together without seeing John eating would constitute fairly strong evidence.  Coming back to our intuitions about the difference between \ref{gen} and \ref{min} versus \ref{plain}, it is clear that the first two are described as emphatic because they are used with stronger standards of evidence. Moreover, higher standards of evidence entail lower standards of evidence, and thus forms used with those higher standards are more informative.

While we have defined standards of evidence in terms of a \emph{given} proposition, in what follows we abstract from this a bit more, assuming that there is a single prior distribution over standards of evidence, $p(t)$, regardless of the proposition being negated. This distribution determines how likely speakers are to have observed facts that warrant a given standard of evidence. Here we model the prior distribution over standards of evidence $T : [0, 1]$ as a \emph{beta distribution}, denoted as $\mathcal{B}(\alpha, \beta)(t)$, where $\alpha, \beta > 0$ are parameters that control the shape of the distribution over states, and the beta function $B(\alpha, \beta)$ serves as a normalizing constant.

\begin{equation}
	\mathcal{B}(\alpha, \beta)(t) = \frac{t^{\alpha -1 }(1-t)^{\beta -1 }}{B(\alpha, \beta)}
\end{equation}
The expected value of a beta distribution over states is given by $\frac{\alpha}{\alpha + \beta}$. As a special case, the uniform distribution over states would correspond to $\mathcal{B}(1,1)(t)$. So the expected standard of evidence given a uniform prior distribution would be $\frac{1}{2}$. A prior distribution $\mathcal{B}(10,10)(t)$ would yield the same expected state, but with decreases variance. That is, the distribution would be more bunched up around the expected state $\frac{1}{2}$. Using a beta distribution allows to flexibly model assumptions about how often speakers have different strengths of evidence.

For example, in what follows we assume that the prior over states is of the general form $\mathcal{B}(\alpha_p,\beta_p)(t)$, where $\alpha_p = 1$ and $\beta_p > 1$. These two parameters allow us to encode facts about the prior distribution over states and thus standards of evidence. For example, as boundedly rational agents, we often have limited evidence for our assertions because we rarely know all of the relevant facts. Fixing $\alpha_p=1$ and letting $\beta_p$ vary is a way to model the fact that we rarely ever know anything beyond the shadow of a doubt.  A larger $\beta_p$ simply corresponds to the prior probability being skewed towards low standards of evidence.  We can visualize the prior distribution over standards of evidence for various values of $\beta_p$ as in Figure \ref{beta}. For $\beta_p = 1$, we have a uniform distribution over standards of evidence. For $\beta_p = 2$ the distribution is skewed towards lower standards of evidence, and as we increase $\beta_p$ further it becomes even more skewed towards these lower standards of evidence. 

\begin{figure}
\begin{center}
	\includegraphics[width=.8\textwidth]{beta-distribution.pdf}
	\caption{Prior distribution over standards of evidence as a beta distribution $\mathcal{B}(\alpha_p, \beta_p)(t)$  with $\alpha_p = 1$ for various values of $\beta_p$.}
	\label{beta}
\end{center}
\end{figure}

Once a speaker's standard of evidence is drawn according to the prior probability distribution her strategy determines what message she sends to the hearer. In the case of the functional cycle, a speaker's strategy consists of a mapping from different standards of evidence to the two forms \emph{\textcolor{red}{$m_{ne}$}} and \emph{\textcolor{blue}{$m_{ne...not}$}}.  Here we take the set of speaker strategies to be the set of mappings from partitions of the state space to the set of messages $S \co \mathcal{P}_n(T) \rightarrow M$, where $\mathcal{P}_n(T)$ is a partition of the state space into $n$ subintervals $t_0 = 0 < t_1 < ... < t_{n-1} < t_n = 1$.\footnote{More precisely, the set of pure speaker strategies consist of all Lebesgue-measurable functions from states to actions. However, the set of pure speaker strategies that partition the state space in this manner are the only ones relevant for our equilibrium analysis in Section \ref{Equilibrium}. See the the appendix for a proof.}  Intuitively, this is simply a way of carving up the state space into discrete contiguous regions that determine the signal sent.  For example, we will deal with the case of two messages $\mathcal{P}_2(T)$, where the speaker  strategy $s$ yields $s(t) = \textcolor{red}{m_{ne}}$ for $t \in [0, \emph{\textcolor{red}{$t_{ne}$}})$ and $s(t) = \textcolor{blue}{m_{ne...not}}$  for $t \in (\emph{\textcolor{red}{$t_{ne}$}}, 1]$.  

So far we have defined the states as standards of evidence and the set of speaker strategies. Together these allow us to offer the following definition of emphatic negation as a highly informative form of negation. Namely, following \cite{skyrms:2010}, we define informativity in terms of the \emph{information gain}, or \emph{Kullback-Leibler divergence} \citeyearpar{kullback-leibler1951divergence} of a message. Where $p(t \mid m)$ is the conditional probability of states given a message and $p(t)$ is the prior probability, the information gained by receiving a message is the following.



\begin{equation}
     KL( m ) = \int_0^1 log\left( \frac{p(t \mid m )}{p(t)}  \right)p(t \mid m ) dt
\end{equation}

If the form $\textcolor{red}{m_{ne}}$ is used at roughly the same rate with all standards of evidence, then the conditional probability of states given the message, $p(t \mid \textcolor{red}{m_{ne}})$, is approximately the prior probability over standards of evidence, $p(t)$. It follows that $KL(\textcolor{red}{m_{ne}})$ will be small since $\frac{p(t \mid \textcolor{red}{m_{ne}})}{p(t)} \approx 1$ and $log(1) = 0$. In other words, upon receiving \textcolor{red}{$m_{ne}$} hearers will know that negation was used, but will not have much additional information about how it was used. That is, very little will have changed from their prior expectations about the speaker's standard of evidence.  In contrast, if the form \textcolor{blue}{$m_{ne...not}$} is overwhelmingly used with higher standards of evidence, then it carries information about how negation is being used. The conditional probability of states given the message is very different from the prior, $p(t \mid \textcolor{blue}{m_{ne...not}}) \neq p(t)$, and thus the message carries substantial information, $KL(\textcolor{blue}{m_{ne...not}}) \gg 0$.  In other words, receiving \textcolor{blue}{$m_{ne...not}$} substantially shifts the hearer's expectations about the evidence the speaker has.   This definition of informativity also offers a natural interpretation of the intuitive relation between the frequency of the forms and bleaching. As \textcolor{blue}{$m_{ne...not}$} increases in frequency relative to \textcolor{red}{$m_{ne}$}, the conditional probability of states given the message necessarily approaches the prior. When \textcolor{blue}{$m_{ne...not}$} is the only form in use it ceases to carry any information about the standard of evidence because it simply cannot shift the prior expectation of hearers.

It should be clear from this definition of emphasis as informativity that there are many ways of emphatically negating a sentence. Simply enumerating negative polarity items should be sufficiently convincing (cf. \citealt{horn:1989}). However, these items do not in and of themselves constitute forms of emphatic negation insofar as they are subject to selectional restrictions. As a point of comparison, at some point in the history of French, \textcolor{blue}{$m_{ne...pas}$}  becomes a form of sentential negation. But, the same cannot be said for \emph{a step} in the history of English. Indeed, \textcolor{blue}{$m_{\text{\emph{n't...a step}}}$} is not a form of emphatic negation in Modern English because \emph{a step} is still restricted to being used to emphatically negate propositions involving verbs of motion.

\ex. \a. I didn't move a step
       \b. \# I didn't eat a step.

Distinguishing ways of emphatically negating propositions from actual forms of emphatic negation has two important consequences. First, at any given point in time a language need not have a form of emphatic negation as we have defined it here. That is, there does not have to be a highly informative form of sentential negation that is free from selectional restrictions. Second, and related to this first point, this does not mean that speakers have no way of emphatically negating propositions. Even without forms of emphatic negation, speakers can still emphatically negate a proposition by using negative polarity items.

Now that we have defined the components of the signaling game that correspond to the speaker's states and strategies, we turn to the hearer.  Once the speaker sends a message, the hearer's response is determined by her strategy. The set of hearer strategies is all potential mappings from the set of messages to the unit interval $R \co M \rightarrow [0,1]$. For each message $m_i$ the hearer takes an action $a_i$. So, for example, $r(\textcolor{red}{m_{ne}}) = \textcolor{red}{a_{ne}}$ is the hearer's response to message \emph{\textcolor{red}{$m_{ne}$}}, and  $r(\textcolor{blue}{m_{ne...not}}) = \textcolor{blue}{a_{ne...not}}$ is the hearer's response to message \emph{\textcolor{blue}{$m_{ne...not}$}}. The crucial point to clarify is what these actions correspond to. In this regard it is useful to note two ways of conceptualizing the purpose of communication. The first is that communication consists of the transmission of information from speaker to hearer. However, as  \cite{franke-etal:2012} note,  if the interests of speakers and hearers diverge, then sharing information presents a paradox. Various mechanisms might serve to bolster this conception of language, but they only succeed insofar as we actually observe the features of communication that they predict.

For example,  communication may be conceived of as a kind of  reciprocal information sharing \citep{trivers1971}. However, such an arrangement would require that speakers monitor their interlocutors to prevent \emph{free-riding}. That is, speakers would be expected to chastise interlocutors for not making substantively informative contributions to a conversation. This is the exact opposite of what we actually observe. Thoughtful listeners are commended for being attentive and polite, whereas long-winded speakers are chastised unless they make substantive contributions. In other words, hearers monitor speakers for informative contributions rather than the other way around. \cite{dessalles2007} notes that we can make sense of this fact if we conceptualize the purpose of communication not just as the transmission of information \emph{per se}, but as the transmission of information in exchange for social status.  When hearers listen to what a given speaker has to say they increase the speaker's status insofar as they choose to listen. That is, given that hearers have a choice this choice acts as a kind of advertisement for speaker's ability to gather and reason about important information (cf. \citealt{gintis2001,dessalles2014}).  The longer and more keenly others listen, the better an indication that a speaker is saying something relevant, and the more social benefits accrue to the speaker. In what follows we take the set of actions available to hearers to constitute a scale, $A \co [0,1]$, which correspond to the amount of time and attention paid to the speaker by the hearer in a given exchange.

If we conceive of communication as the exchange of information for social status, the final part of the signaling game to specify is the preferences of speakers and hearers over the exchange rate. Intuitively, hearers want the best return on their investment of time and attention in the form of information about the world. Hearers want to spend as much attention as necessary to gain information regarding a given standard of evidence, no more and no less.  In contrast, speakers are biased towards accruing social status, regardless of the actual standard of evidence. The preferences of speakers and hearers diverge depending on the strength of this speaker bias. 

We can model the difference in the preferences of speakers and hearers  using the following utility functions (cf. \citealt{crawford-sobel:1982}). In this case, $U_S$ represents the preferences of speakers and $U_R$ represents the preferences of hearers. For a given state, $t$, the speaker's strategy $s$ determines a message, $m = s(t)$, and the hearer's strategy $r$ determines an action $a = r(m)$. 

 \begin{equation}
      U_S(t, a) = 1 - (a - t - (1-t)b)^2
      \label{U_S}
 \end{equation}
 
 \begin{equation}
      U_R(t, a) = 1 - (a - t)^2
      \label{U_R}
 \end{equation}
To see how these utility functions capture the preferences of speakers and hearers, consider the hearer utility function, $U_R$. Remember that this utility function is meant to encode the preferences of hearers. Above, we noted that hearers prefer an even exchange of informative evidence for social status. We can capture this fact in a straightforward manner by using a utility function that is maximized where the state and action are equivalent, $a = t$. In other words, this utility function simply represents the fact that hearers prefer to invest the appropriate amount of time and attention given the standard of evidence. 

In contrast, speakers prefer that the amount of attention invested exceeds the level warranted by the standard of evidence. To see how this is represented, consider the speaker utility function, $U_S$. As we noted above, speakers prefer hearers to invest more time and attention than is warranted by the information they have to share.  The degree of speaker bias is captured by the parameter $b$. For $b > 0$ the action that maximizes the speaker's utility, $a = t + (1-t)b$, is always greater than or equal to the actual state.  Note that this utility function represents the speaker's preference for actions higher than a given state, but also guarantees that speakers can only ever prefer possible actions. We should also note that the message sent by the speaker has no bearing on utility. There is no cost to using message over another; talk is cheap.

Now, to clarify, there are many different utility functions for speakers and hearers that could be used to represent the preferences we have described above. For example, instead of squaring the distance between states and actions, we could take the absolute value, or the fourth power, or any even integer. Likewise, we could pick any function to represent speaker's preferences for higher actions that also guarantees speakers only ever prefer possible actions. We have chosen these particular forms because they are simple and variants of them are well studied. This means that we have an interesting point of comparison to the prior literature, as well as straightforward functional forms to work with.

At this point we pause to summarize the structure of the interaction between speakers and hearers that we take to underly the use of emphatic negation, and thus the functional cycle. The speaker observes some facts about the world which corresponds to a standard of evidence, and then chooses a message from the forms \emph{\textcolor{red}{$m_{ne}$}} and \emph{\textcolor{blue}{$m_{ne...not}$}} according to a strategy. In response to this message, the hearer takes an action which corresponds to investing some amount of time and attention with the speaker. Hearers prefer an equal exchange of information for the status granted by this attention, but speakers are biased towards a higher rate of attention for a given standard of evidence. Note that this description of the signaling game does not in and of itself predict the behavior of speakers and hearers. To do so, we need to supply a \emph{solution concept} that we can use to determine the equilibria of the game.


\section{Equilibria of the signaling game}
\label{Equilibrium}

With the components of the game defined, we now turn to analyzing its equilibria in order to understand the conditions for the functional cycle. In particular, we want to know what speaker and hearer strategies are \emph{evolutionarily stable strategies} \citep{maynard-smith1982}, behaviors that are resistant to change, and which strategies are not. We begin by defining the expected utility of speaker and hearer strategies given the prior probability over states, and then determine the evolutionarily stable strategies of the game based on the expected utilities. Finally, we note two important points regarding the relationship between speaker bias and the functional cycle.  First, if speaker bias is sufficiently large, then only a single message can be used in equilibrium. Second, this single message equilibrium is not evolutionarily stable since it can always be disrupted by the introduction of an appropriately conditioned signal. This means that if speakers are sufficiently biased, then the functional cycle can always be induced by the introduction of new forms of emphatic negation.

For a given interaction, a speaker observes some facts about the world corresponding to a standard of evidence, uses either a plain or emphatic form of negation, and the hearer responds by paying a certain amount of attention to the speaker. While we can describe a given outcome in these terms, we are  interested in how well speaker and hearer strategies do on average, we want the expected utilities of speaker and hearer strategies. For a pair of speaker and hearer strategies $s$ and $r$ these are given by the following, again where $m = s(t)$, and $a = r(m)$.

\begin{equation}
     E[U_S(t, a)] = \int_0^1 \left( 1 -(a - t - (1-t)b)^2 \right)p(t)dt
\end{equation}

\begin{equation}
      E[U_R(s, a)] = \int_0^1 \left( 1 -(a - t)^2 \right) p(t) dt
\end{equation}

With the expected utilities defined, we might ask whether particular strategies constitute \emph{evolutionarily stable strategies} \citep{maynard-smith1982}. These evolutionarily stable strategies correspond to speaker and hearer behaviors that are resistant to change, and meet the Gricean \citeyearpar[29]{grice1975} criterion of being reasonable to follow rather than abandon. For asymmetric games, such as signaling games where players have specific roles like speaker and hearer, the evolutionarily stable strategies correspond to the \emph{strict Nash equilibria} of the game \citep{selten:1980}.  As a point of reference, a pair of speaker and hearer strategies is a Nash equilibrium if neither speaker nor hearer would do better by unilaterally changing behavior. Such an equilibrium is \emph{strict} if both speaker and hearer would do worse by unilaterally changing behavior.  

Since strict Nash equilibria are behaviors that jointly maximize the utility functions, then they can be determined by solving a system of partial derivatives of the utility functions. That is, we find the point \emph{\textcolor{red}{$t_{ne}$}} where the speaker partitions the state space and the actions \emph{\textcolor{red}{$a_{ne}$}} and \emph{\textcolor{blue}{$a_{ne...not}$}} taken by the hearer, which jointly maximize the expected utilities of both speakers and hearers.

\begin{equation}
	\frac{\partial E[U_S(t, a)]}{\partial \emph{\textcolor{red}{$t_{ne}$}}} =  0 \\
\end{equation}

\begin{equation}
	\frac{\partial E[U_R(t, a)]]}{\partial \emph{\textcolor{red}{$a_{ne}$}}} = 0 \\
\end{equation}


\begin{equation}
	\frac{\partial E[U_R(t, a)]}{\partial \emph{\textcolor{blue}{$a_{ne...not}$}}} = 0
\end{equation}

For calculating the evolutionarily stable strategies we assume a prior probability over standards of evidence $p(t) = \mathcal{B}(1,2)(t)$.  What we are really interested in, however, is the dependence of evolutionarily stable strategies, if they exist, on speaker bias. In Figure \ref{ess-plot} we plot the Nash equilibria strategies of speakers and hearers as a function of speaker bias.\footnote{See the supplementary material for the details of calculating these equilibria and the rest of the code used in the analysis: \http{github.com/christopherahern/CCTJC} We can see this prior probability distribution in Figure \ref{beta}. Also see the appendix for a proof that only pure speaker and hearer strategies as we have described them above can constitute components of evolutionarily stable strategies.}  In what follows we evaluate whether or not given Nash equilibria of the game are strict and thus evolutionarily stable strategies. The horizontal axis of Figure \ref{ess-plot} represents the degree of speaker bias, where $b=0$ indicates the case where speakers are interested in a fair exchange of information for attention and $b > 0$ indicates an increasing bias towards higher rates of exchange. The vertical axis represents the actions and states taken by speakers and hearers in equilibrium for a given amount of bias. The solid line indicates the point at which speakers partition states, below which speakers send \emph{\textcolor{red}{$m_{ne}$}} and above which they send \emph{\textcolor{blue}{$m_{ne...not}$}}. The dashed lines indicate the hearer response \emph{\textcolor{red}{$a_{ne}$}} and \emph{\textcolor{blue}{$a_{ne...not}$}}.  

We can interpret this figure by fixing a value of $b$ and examining how speakers partition the state space and how hearers respond to the messages.  That is, we can imagine a scenario where speakers are biased to a certain degree and examine how speakers and hearers would behave. For example, when $b=0$, speakers partition states at $t_1 = .3819$, sending \emph{\textcolor{red}{$m_{ne}$}} for $t \in (0, .3819)$ and \emph{\textcolor{blue}{$m_{ne...not}$}} for $t \in (.3819, 1)$. In response, hearers take actions \emph{\textcolor{red}{$a_{ne}$}} $=.1759$ and \emph{\textcolor{blue}{$a_{ne...not}$}} $ =.5879$. 

\begin{figure}
\begin{center}
	\includegraphics[width=.8\textwidth]{ess-plot.pdf}
	\caption{Nash equilibrium solutions for two messages for values of bias with prior distribution $p(t) = \mathcal{B}(1,2)(t)$}
	\label{ess-plot}
\end{center}
\end{figure}

There are two important things to note about these results. First, if speaker bias is sufficiently large, then only a single message is used in a Nash equilibrium. That is, for all $b > \frac{1}{6}$ only $\textcolor{blue}{m_{ne...not}}$ will be used in equilibrium.  When speaker bias is this large the form carries no information about the standard of evidence, and the best response for hearers is the action that corresponds to the expected value of the prior. In this case, we see that for sufficiently large speaker bias the best response is \emph{\textcolor{blue}{$a_{ne...not}$}} $ = \frac{1}{3}$. Second, if speaker bias is sufficiently large, this single message equilibrium is not a strict Nash equilibrium and thus is not evolutionarily stable. That is, it can be invaded by strategies corresponding to other behaviors. To see why this is the case note that if only $\textcolor{blue}{m_{ne...not}}$ is used by speakers, then hearers' response to $\textcolor{red}{m_{ne}}$ or any new message $m'$ is free to vary without affecting the expected utility of speakers or hearers, and thus the single message equilibrium is not evolutionarily stable.  

In a certain sense the fact the single message equilibrium is not evolutionarily stable is not particularly informative. However, we can reason about what kinds of behaviors would destabilize this equilibrium. It should be clear that speakers and hearers would prefer to be able to use more than one signal. For example, when speakers have a high standard of evidence they would prefer that hearers listen intently to them, and indeed hearers would prefer to listen intently when speakers do have a high standard of evidence. If speakers and hearers could agree on a particular message to use with high standards of evidence, then both would be better off for it. That is, the single message equilibrium could always be disturbed by the introduction of a new form of emphatic negation; a single message equilibrium is not \emph{neologism-proof} \citep{farrell:1993}.

Now, it should be clear that there is no sense in which a population of speakers and hearers could explicitly agree on a new form to use in particular situations. However, the notion of neologism-proofness admits of a kind of evolutionary interpretation as well \citep[526]{farrell:1993}. Namely, speakers \emph{happen} to use a particular form of emphatic negation with high standards of evidence and hearers \emph{happen} to pay more attention to that form. Over time speakers and hearers implicitly coordinate on a new form of emphatic negation to signal a high standard of evidence. This has interesting implications for the functional cycle: it is not just the existence of new forms of emphatic negation that is relevant. A form has to be available, but speakers and hearers also have to \emph{happen} to coordinate on  using it. For example, for speakers and hearers to coordinate on using a new form of emphatic negation, the form itself might need to be used with sufficient frequency. There may be some critical frequency threshold below which the use of different forms of emphatic negation is subject to stochastic fluctuations, but above which the new emphatic form sets the stage for the functional cycle.  We return to this point in the next section.

In this section we determined the evolutionarily stable strategies of the population. We reasoned about the stability of those states, but the reasoning we used was essentially static. That is, we reasoned about what would happen if we started at a particular state, but not whether we would ever reach that state in the first place. Importantly, this kind of reasoning does not allow us to understand how a population evolves in a particular historical change. We must posit a process that underlies how speakers and hearers interact and respond to each other. Doing so allows us to examine how different degrees of speaker bias impact the trajectory of meaning. 


\section{The dynamics of the functional cycle}
\label{Dynamics}

In the previous section we examined the equilibrium properties of speaker and hearer strategies in the signaling game as we varied how biased speakers are in the exchange of information for attention and status. If speakers are sufficiently biased, then the functional cycle can be set in motion with the introduction of a new form of emphatic negation. In this section we turn to the details of the motion of the functional cycle by providing \emph{evolutionary game dynamics} that define how a population of speakers and hearers changes over time \citep{hofbauer-sigmund1998}. We begin by discussing the \emph{replicator dynamics} \citep{taylor-jonker:1978} as an evolutionary game dynamics for studying changes in meaning. We then turn to data from the functional cycle in a parsed corpus of Middle English. Finally, we fit a dynamic model to this data and assess, via model, comparison the relative support we have in favor of positing a bias on  the part of speakers.

The replicator dynamics were originally introduced as an explicitly dynamic model of biological replication. The fundamental intuition underlying them is that strategies that have a higher than average expected utility should increase in prevalence in a population, and that strategies with lower than average expected utility should decrease. In the biological context, utility is interpreted in terms of reproduction and strategies with higher expected utilities have more offspring than average. However, this simple idea has since been shown to have deep connections with some of the most widely-studied models of learning. In particular, \cite{borgers-sarin1997} prove that if agents interact frequently and change their behavior slowly then the asymmetric replicator dynamics are equivalent to the expected behavior of agents playing an asymmetric game while learning according to a simple form of learning (cf. \citealt{bush-mosteller1955, sutton-barto1998}). That is, if speakers and hearers tend to do things more if they yield higher utility, then their expected behaviors can be modeled by the replicator dynamics. 

In what follows, we will use the replicator dynamics to model the functional cycle, so before moving on to defining the dynamics themselves, it is important to clarify what they are a model of. There are two points to be made. First, while we refer to speaker and hearer populations, we assume that each individual person acts as a part of each population. So, when we simply refer to a population we mean a population of individuals; when we refer to the speaker population we mean how the population of individuals behave as speakers; and when we refer the hearer population we mean how the population of individuals behave as hearers. Second, the replicator dynamics model the expected or \emph{mean dynamics} of signaling behavior in an entire population. They are a deterministic model of change, whereas language change itself is a stochastic process allowing for variation between individuals over time.  However, when it comes to language change, the mean behavior of the population is a remarkably close approximation of individual behavior \citep{kroch1989}. So, we will discuss the dynamics \emph{as if} they characterized the knowledge of all individuals in a population that are born, adjust their use of the different forms according to the dynamics, and eventually die. 

To construct the game dynamics we use a discretized set of states and actions for speakers and hearers. That is, for some $n$, we treat the set of states $T \co \{t_0, ..., t_n \}$ and actions $A \co \{a_0, ..., a_n \}$, where $t_i = a_i = \frac{i}{n}$. In this case, we use one hundred states and actions, $n=100$, and use \emph{beta-binomial} distributions over these states and actions as an approximation to the beta distribution.\footnote{For our purposes, this discretization allows us a tractable means of simulating and fitting the dynamics to data. Some analytical results can be derived regarding the  details of the replicator dynamics in continuous strategy spaces, but are more complicated  and ultimately rely on numerical simulations for assessing stability. See  \cite{oechssler2002,jager2011} for relevant discussion. See the appendix for a more detailed discussion of the formulation of the dynamics as well as a visual comparison of the beta and beta-binomial distributions.} Here we use the discrete-time version of the behavioral replicator dynamics suggested by \cite{hofbauer-huttegger2015}, which treats each state as its own population in which messages compete with each other, and each message as its own population in which actions compete with each other.  The dynamics define the evolution of two matrices, $\mathbf{X}$  and $\mathbf{Y}$,  which correspond to speaker and hearer populations respectively: $\mathbf{X}_{ij}$ is the probability of speakers sending message $m_j$ in state $t_i$, and $\mathbf{Y}_{ij}$ is the probability of hearers taking action $a_j$ in response to message $m_i$.

From one point in time to the next, these matrices change according to the rules in \eqref{speaker} and \eqref{hearer}. With slight abuse of notation, the probability of using message $m_j$ in state $t_i$ at the next point in time is determined by the current probability of doing so, $\mathbf{X}_{ij}$, and the ratio between the expected utility of doing so given the hearer response, $E[U_S(\mathbf{X}_{ij}, \mathbf{Y})]$, and the average expected utility in state $t_i$. If sending message $m_j$ does better than average, then it will be used more in state $t_i$ because the ratio will be greater than one.

\begin{equation}
     \mathbf{X}_{ij}' = \mathbf{X}_{ij}\frac{E[U_S(\mathbf{X}_{ij}, \mathbf{Y})]}{\sum_j \mathbf{X}_{ij} E[U_S(\mathbf{X}_{ij}, \mathbf{Y})]}
     \label{speaker}
\end{equation}

\begin{equation}
     \mathbf{Y}_{ij}' = \mathbf{Y}_{ij}\frac{E[U_R(\mathbf{X}, \mathbf{Y}_{ij})]}{\sum_j \mathbf{Y}_{ij} E[U_R(\mathbf{X}, \mathbf{Y}_{ij})]}
          \label{hearer}
\end{equation}
The same holds for the use of action $a_j$ in response to message $m_i$. Again, if response $a_j$ does better than average, then its use will increase at the next point in time. Together  
\eqref{speaker} and \eqref{hearer} determine how the populations change over time.

Now, defining the game dynamics allows us to consider the effect of speaker bias on the functional cycle in the abstract, but we are really interested in how the resulting model can be applied to the actual historical trajectory of negation. In particular, we are interested in what happens when we fit the model to data from the history of negation in English. Towards this end, we fit the resulting model to 5,475 tokens of negative declaratives drawn from the second edition of the Penn Parsed Corpus of Middle English \citep{ppcme2}.\footnote{Following \cite{wallage2008} and \cite{ecay-tamminga2015}, we use negative declarative tokens, but exclude contracted forms, negative concord, and cases that appear to be constituent negation, among other cases that pattern in a substantially different manner. The code for generating and processing the queries can be found at: \http{github.com/christopherahern/jespersens-cycle-middle-english}} The data are shown in Figure \ref{neg-three-plot}. The horizontal axis represents years, spanning Middle English from 1100 to 1500 CE. Each document is represented by three circles which correspond to the different forms of negation. The size of the circles represents the total number of tokens in the document. The vertical placement of the circles represents the proportion of the forms that fall into each category. Take the circles that represent the document at 1200.  There are 178 tokens in the document, 139 of these tokens are \textit{\color{red} ne} so the red circle is at 0.78, 39 of these tokens are \textit{\color{blue} ne...not} so the blue circle is at 0.22, and none of these tokens are \textit{\color{green} not} so the green circle is at 0.00.  Loess curves are fit to the proportions of the different forms in documents over time.\footnote{These loess curves are \emph{not} being proposed as a model of change and they should not be interpreted as such. Absent any particular mechanism of change, there is no sense in which we could interpret what such a model would even mean. We return to this point below.} We see the transition from \textit{\color{red} ne} to \textit{\color{blue} ne...not} starting around the 12th century. In 1350 \textit{\color{blue} ne...not} is the majority form, but is quickly replaced by \textit{\color{green} not} by the turn of the 15th century.


\begin{figure}
\centering
     \includegraphics[width=.75\textwidth]{neg-plot.pdf}
\caption{Proportion of \textit{\color{red} ne}, \textit{\color{blue} ne...not}, and \textit{\color{green} not}  in negative declaratives over time in the PPCME}
\label{neg-three-plot}
\end{figure}


Our goal is to model the functional cycle in English, which means that we want to offer an explanation of the observed transition from \textit{\color{red} ne} to \textit{\color{blue} ne...not}. This is made easier by the fact that we have a reasonably large annotated corpus and tools for extracting tokens in the relevant time frame. This is obviously also complicated by the subsequent rise of \textit{\color{green} not}, which represents the second transition in the formal cycle.  As we see it there are three options to address this fact. One option would be to only compare \textit{\color{red} ne} and \textit{\color{blue} ne...not}, excluding \textit{\color{green} not} entirely from our analysis. However, this would run the risk of attributing too much to noisy fluctuations after a certain point in time when \textit{\color{red} ne} and \textit{\color{blue} ne...not} combined cease to be the majority of forms. Indeed, after 1350 \textit{\color{red} ne} becomes more frequent than \textit{\color{blue} ne...not} again. Yet, it does not seem reasonable to take this as evidence for a reversal of the functional cycle.   Another option would be to only compare \textit{\color{red} ne} and \textit{\color{blue} ne...not} up until 1350. This would allow us to estimate parameters of the model, but, if we know that the trajectory of \textit{\color{red} ne} versus \textit{\color{blue} ne...not} differs dramatically after 1350, then we should not give too much credence to the model fit to this data. A third option, is to treat tokens of post-verbal negation \emph{as if} they were tokens of bipartite negation. That is, we can lump \textit{\color{blue} ne...not} and \textit{\color{green} not} together for the purposes of fitting the model. One rationale for doing so stems from Jespersen's \citeyearpar{jespersen:1917} observation that the preverbal element \emph{ne} is phonetically light and more likely to be misperceived or not perceived at all. If instances of post-verbal \textit{\color{green} not}  arise from misperception of the bipartite form \textit{\color{blue} ne...not}, then we would expect them to be used at approximately the same rate with different standards of evidence.  If this is the case, then we would expect both of these more informative forms to increase in use at approximately the same rate.  That is, the mechanism that leads to the second transition of the formal cycle would be orthogonal to the mechanism that drives the functional cycle. 

Here we adopt this third option and assume that the mechanism driving the second transition of the formal cycle is indeed independent of the functional cycle. However, we should also note that evaluating the validity of this choice depends on fitting the model to time series of the functional cycle from languages where the second transition of the formal cycle does not complicate the analysis. Other languages, such as French \citep{martineau-mougeon2003}, Dutch \citep{burridge1993}, and Flemish \citep{vanderAuwera-neuckermans2004}, would offer an important point of comparison.  If fitting the model to time series of the functional cycle from these languages yields similar results to those presented here, then our treatment of \textit{\color{blue} ne...not} and \textit{\color{green} not} would seem reasonable. If the results were substantially different, then we would need to revisit this assumption.   Treating \textit{\color{blue} ne...not} and \textit{\color{green} not} as the same yields the results shown in Figure \ref{func-plot}, where the horizontal axis represents the same time span, but the vertical axis represents the proportion of bipartite \textit{\color{blue} ne...not} and post-verbal  \textit{\color{green} not} versus pre-verbal \textit{\color{red} ne}. Again, each circle represents an individual document whose size corresponds to the number of tokens.


\begin{figure}
\centering
     \includegraphics[width=.8\textwidth]{func-plot.pdf}
\caption{Proportion of emphatic \textit{\color{blue} ne...not} and \textit{\color{green} not}  versus  \textit{\color{red}  ne} in negative declaratives over time in the PPCME}
\label{func-plot}
\end{figure}

Since, we cannot evaluate the standard of evidence for any given token of negation, the trajectory of forms in Figure \ref{func-plot} constitutes the data available to fit our model. To actually fit the model, we need to specify its parameters. In particular, we need to define the initial state of how speakers use the different forms and how hearers respond to them. From there we can simulate the dynamics and adjust these parameters to find the most likely parameters given the data.  In fact, we have quite a bit of information regarding what the initial state of the functional cycle actually is. That is, we know that \textit{\color{blue} $m_{ne...not}$}  is fairly infrequent and largely restricted to higher standards of evidence. Likewise, we know that hearers' response to \textit{\color{blue} $m_{ne...not}$}  is also largely restricted to actions corresponding to higher standards of evidence. We can translate this information into conditions on the initial states of the speaker and hearer populations.


Regarding the initial state of the speaker population, the incoming form \textit{\color{blue} $m_{ne...not}$} should be used infrequently, and when it is used it should be used almost exclusively with higher standards of evidence. There are many ways of implementing these initial conditions. Here we assume that that the use of  \textit{\color{blue} $m_{ne...not}$} given states is determined by a beta-binomial distribution with $n=100$ states $p(\textcolor{blue}{m_{ne...not}} \mid t ) = \mathcal{B}(\alpha_{s}, 1)(t)$.\footnote{In what follows we use beta-binomial distributions, but omit the number of states $n=100$ from the notation.}Notice that \textcolor{blue}{$m_{ne...not}$} is used with a very low probability in all states, $p(\textcolor{blue}{m_{ne...not}} \mid t_i)$ is small for all states $t_i$. For example, in the case where $\alpha_s = 1$, and \textcolor{blue}{$m_{ne...not}$} is used uniformly across all states, the probability that \textcolor{blue}{$m_{ne...not}$} is used in any state is $p(\textcolor{blue}{m_{ne...not}} \mid t) = \frac{1}{100}$. Also note that using this distribution and the prior probability over states, we can calculate the posterior probability of standards of evidence given the message according to Bayes' rule, $p(t \mid \textcolor{blue}{m_{ne...not}}) = \frac{p(\textcolor{blue}{m_{ne...not}} \mid t ) p(t)}{\int_0^1 p(\textcolor{blue}{m_{ne...not}} \mid t ) p(t)dt}$. As $\alpha_{s}$ increases, this posterior distribution shifts towards higher and higher standards of evidence. In contrast, \textit{\color{red} $m_{ne}$} is the default form and does not carry much information above and beyond the prior. Since \textit{\color{blue} $m_{ne...not}$} is used infrequently, this means that \textit{\color{red} $m_{ne}$} is used almost evenly across all standards of evidence. This means that it roughly satisfies the following distribution $p(t \mid \textit{\color{red} $m_{ne}$}) \approx \mathcal B (1, \beta_p)(t)$. Here $\beta_p$ is the shape parameter of the prior, which determines its skew towards lower standards of evidence as we showed in Figure \ref{beta}. The last parameter to be fit for speakers is the degree of speaker bias $b$. 

In all, then, we have three parameters, $\alpha_{s}$, $\beta_p$, and $b$, to fit to model the initial state of and subsequent changes to the speaker population. To put these in perspective, the parameter $\alpha_{s}$ allows us to capture the fact that \textit{\color{blue} $m_{ne...not}$} is largely restricted to the highest standards of evidence at the start of the functional cycle, and is thus an emphatic form of negation. The parameter $\beta_p$ allows us to capture the fact that we are boundedly rational agents with limited informational resources and thus often find ourselves with low standards of evidence. The parameter $b$ allows us to capture the fact that despite often having very little information as speakers, we prefer that our hearers pay attention to us. In fact, these three parameters are sufficient since we can reasonably define the initial state of the hearer population with them as well. Intuitively, we want hearers to have reasonably accurate responses to the two forms. This can be achieved by ensuring that the probability with which hearers take an action is directly proportional to its expected utility.\footnote{Thanks go to Michael Franke for this astute suggestion for modeling the initial state of the hearer population.} So,  $p(a \mid \textit{\color{red} $m_{ne}$}) \propto \sum_i p(t_i \mid \textit{\color{red} $m_{ne}$}) U_r(t_i, a)$ and $p(a \mid \textit{\color{blue} $m_{ne...not}$}) \propto \sum_i p(t_i \mid \textit{\color{blue} $m_{ne...not}$}) U_r(t_i, a)$. To guarantee that these are probabilities, we simply normalize these expected utilities.

Fitting the dynamic model to the time series data from the functional cycle involves the following. First, we  specify the parameters of the initial condition, $\alpha_{s}$, $\beta_p$, $b$. Then we  simulate the replicator dynamics for the four hundred years from 1100 to 1500, calculating the probability of \textcolor{blue}{$m_{ne...not}$} at each year as $p(\textcolor{blue}{m_{ne...not}}) = \sum_t p(\textcolor{blue}{m_{ne...not}} \mid t) p(t)$.\footnote{See the supplementary materials for code and details. We should also note that the replicator dynamics evolve in abstract time units. Here we assume that each of these abstract time units corresponds to a year. However, determining the appropriate ratio between abstract units and actual time is an important question we leave for future research.}  The likelihood of the parameters given the data in a given year is given by the binomial likelihood function, where $N$ is the total number of tokens, and $n$ is the number of \textcolor{blue}{$m_{ne...not}$} tokens.

\begin{equation} 
\binom{N}{n} p(\textcolor{blue}{m_{ne...not}})^{n}(1-p(\textcolor{blue}{m_{ne...not}}))^{N - n}
\end{equation}
The likelihood of the parameters given the data is then the product of the likelihoods for the individual years. Alternatively, the log-likelihood of the parameters given the data is the sum of the log-likelihoods for the individual years. We calculate the log-likelihood for a wide range of parameters and calculate the likelihood of the parameters. Finally, we take the best parameters from this set and iteratively change them in order to maximize the log-likelihood.

The parameters of the model that maximize the log-likelihood, $log\mathcal{L} = -313.574$, are the following.  Regarding the initial use of the incoming form, $\hat{\alpha}_{s} = 3.065$, $\hat{\beta}_p =  9.852$. As a point of comparison, consider these parameters in terms of Figure \ref{beta}. If we keep $\beta=1$ and increase $\alpha_s$ then the distribution is more and more skewed to the right. The fact that $\hat{\alpha}_{s} = 3.065$ simply means that at the beginning of the functional cycle speakers are more likely to use \textcolor{blue}{$m_{ne...not}$} with higher standards of evidence. The fact that $\hat{\beta}_p =  9.852$ simply means that the prior distribution over standards of evidence is highly skewed towards low standards of evidence. That is, we very rarely find ourselves in a position to make strong claims, we are often uncertain about the world. The fact that $\hat{b}=0.300$ means that speakers are fairly biased towards preferring more attention than a given standard of evidence warrants. We can visualize these results by simulating the replicator dynamics using these parameters, and showing the use of \textcolor{blue}{$m_{ne...not}$} over time as in Figure \ref{m2-sol}. The horizontal axis represents years and the vertical axis represents the probability of use of \textcolor{blue}{$m_{ne...not}$} over time.  

\begin{figure}
\centering
     \includegraphics[width=.8\textwidth]{p-ne-not.pdf}
\caption{Predicted probability of \textit{\color{blue} $m_{ne...not}$} over time for fitted model of functional cycle.}
\label{m2-sol}
\end{figure}

We can also examine the information carried by the emphatic form over time as in Figure \ref{m2-meaning}. The horizontal axis represents standards of evidence and the vertical axis represents the conditional probability of standards of evidence given that \emph{\textcolor{blue}{$m_{ne...not}$}} was used. We show this conditional probability at the turn of each century from 1100 to 1500 as the functional cycle proceeds, where the dashed line indicates the prior probability over standards of evidence.  The initial meaning of the incoming emphatic form at 1100 is represented by the thickest curve. This conditional probability of states given \emph{\textcolor{blue}{$m_{ne...not}$}} is distinct from the prior, with a much higher expected value. At this point in time the incoming emphatic form carries information. At 1200, the conditional probability of states given \emph{\textcolor{blue}{$m_{ne...not}$}} is also distinct from the prior. As time goes on, \emph{\textcolor{blue}{$m_{ne...not}$}}  shifts towards lower standards of evidence as the form increases in frequency.  By 1300 \emph{\textcolor{blue}{$m_{ne...not}$}} still carries information, but not much and by 1400 the probability of states given \emph{\textcolor{blue}{$m_{ne...not}$}} is almost identical to the prior. We represent this loss of information by the thickness of the line. By 1400, \emph{\textcolor{blue}{$m_{ne...not}$}} is hardly distinguishable from the prior both visually and in terms of information.

\begin{figure}
\centering
     \includegraphics[width=.8\textwidth]{pt-ne-not.pdf}
\caption{The conditional probability of states given \textit{\color{blue} $m_{ne...not}$} at the turn of each century.}
\label{m2-meaning}
\end{figure}

Interestingly, the conditional probability of states given \emph{\textcolor{blue}{$m_{ne...not}$}}  at 1100 is not the rightmost curve in Figure \ref{m2-meaning}.\footnote{Here we use the form of distributions the KL-divergence which is defined for discrete distributions. For a given message the KL-divergence is $KL( m ) = \sum_i log\left( \frac{p(t_i \mid m )}{p(t_i)}  \right)p(t_i \mid m )$} Indeed, if we plot the information carried by \emph{\textcolor{blue}{$m_{ne...not}$}}  over time as in Figure \ref{kl} we see that the information carried by the form actually increases up until the midpoint of the 12th century. From this point on it strictly decreases as \emph{\textcolor{blue}{$m_{ne...not}$}} increases in frequency and is bleached of its emphasis. A reasonable interpretation of this trajectory is that during the first half of the 12th century speakers and hearers are coordinating on \emph{\textcolor{blue}{$m_{ne...not}$}} as a new form of emphatic negation. At some point though the increase in the informativity is halted as the new form starts to be exploited by speakers with lower standards of evidence. From this point on, we see the functional cycle unfold as an inflationary process. If we take the empirical proportions of forms in documents as a rough estimate, then the critical threshold for a new form of emphatic negation to set off the functional cycle may lie somewhere between 0.1 and 0.2. The actual value of this kind of frequency threshold would have to be investigated further by comparison to the functional cycle in other languages. However, it also makes a novel prospective prediction about when a language might undergo a functional cycle. Namely, if a form of emphatic negation reaches the critical threshold, then we would expect it to set off a functional cycle.


\begin{figure}
\centering
     \includegraphics[width=.8\textwidth]{kl-plot.pdf}
\caption{The information carried by \textit{\color{blue} $m_{ne...not}$} over time as measured by the KL-divergence of \textit{\color{blue} $m_{ne...not}$}.}
\label{kl}
\end{figure}

Now, a reasonable question to ask at this point is whether positing a bias on the part of speakers is justified. We can assess this by comparing the full dynamic model to a simplified model where $b=0$ and the other parameters are free to vary. This allow us to assess the relative contribution of positing a bias on the part of speakers.  The parameters of this simplified model that maximize the likelihood are $\hat{\alpha}_{s} = 50.378$, $\hat{\beta}_p =   0.064$ and yield a log-likelihood of $log\mathcal{L} = -885.000$. There are two distinct problems with these results. First, contrary to our discussion earlier, the simplified model has a prior probability over standards of evidence where speakers are almost always absolutely sure of what they say. Second, as we would expect from the equilibrium analysis in Section \ref{Equilibrium}, in the absence of a speaker bias  the functional cycle will not go to completion. We can visualize the results of the simplified model by simulating the replicator dynamics for these parameters, as in Figure \ref{m2-sol-simplified}. As we can see, the functional cycle does not go to completion before 1500. Indeed, under the simplified model, where $b=0$, the use of the \textcolor{blue}{$m_{ne...not}$} would still be approaching its equilibrium value at the present day. We take this as convincing evidence for the full model where speakers exhibit a non-zero bias $b > 0$.

\begin{figure}
\centering
     \includegraphics[width=.8\textwidth]{p-ne-not-simplified.pdf}
\caption{Predicted probability of \textit{\color{blue} $m_{ne...not}$} over time for simplified model of functional cycle where $b=0$.}
\label{m2-sol-simplified}
\end{figure}

We can also compare the full and simplified models quantitatively by examining the trade-off between model fit and model complexity using Akaike's Information Criterion \citep{akaike1974}, where $k$ is the number of parameters  in the model. 
 
\begin{equation}
	AIC = 2k - 2\mathcal{L}
\end{equation}
Since we can always increase the log-likelihood of the model by adding more parameters, $AIC$ penalizes the addition of more parameters. Thus, smaller values indicate better models.  For the full model  $AIC_{full} = 633.148$ and for the simplified model $AIC_{simplified} = 1774.001$. 

If positing a bias on the part of speakers leads to a substantially better fit of the model despite being more complex, then the difference between the two models,  $\Delta AIC = AIC_{simplified} - AIC_{full} = 1140.853$, should be large. To put this in context for the case of our two models, the probability of the simplified model being the better model given the data is the following \citep[74-79]{burnham2003}.\footnote{This is the \emph{Akaike weight} of the simplified model, which generally speaking has a larger $AIC$. These weights are normalized to sum to one and can thus be interpreted as probabilities. Note also that since the full and simplified model are nested, we can also perform a Likelihood Ratio test. The test statistic $D = 2(\mathcal{L}_{full} - \mathcal{L}_{simplified}) = 1142.853$ is approximately $\chi^2$ distributed with one degree of freedom ($p < .000$). Thus we can reject the null hypothesis of the simplified model being correct within a null hypothesis testing framework.}

\begin{equation}
	\frac{1}{1 + e^{\frac{1}{2}\Delta AIC}}
\end{equation}
It should be clear that the chances of the simplified model actually being the better model are negligible. To put it in perspective, this probability is roughly the magnitude of flipping a fair coin eight hundred times and it only ever coming up heads. We take this as convincing evidence in favor of the full model that posits a bias on the part of speakers.

\section{Conclusion}
\label{Conclusion}

At this point is useful to summarize the steps we have taken. We started off by distinguishing between the formal and functional Jespersen cycles, and focused on the functional cycle as a kind of inflationary process. At the start of the cycle, a new form of emphatic negation is introduced, but loses emphasis as it increases in frequency. We modeled these facts by defining a signaling game between speakers and hearers. Speakers observe some facts about the world, which determine a standard of evidence. Speakers signal this information using different forms of negation, and hearers respond by paying varying amounts of attention and thus granting social status to the speaker. While hearers want a fair exchange of information for status, speakers are biased towards gaining more status for less information. Emphatic forms, which we defined as highly informative and free from selectional restrictions, are exploited by speakers due to this conflict of interests. But, as emphatic forms increase in frequency, they cease to be highly informative and hearers discount the amount of attention paid to them. We fit a dynamic model of this interaction to time series data from the functional cycle in a corpus of Middle English. We found that positing a bias on the part of speakers is supported by model comparison.

There are two points to be made regarding these results. The first main point is that while we have only compared two nested models, the same comparisons could be made to different variants of the model: we could allow for differences in bias or priors over standards of evidence across different contexts, across speakers, or even across time. For example, we abstracted from states as standards of evidence for a particular proposition. If we had empirical evidence for different standards of evidence for different kinds of propositions, along with their relative frequencies we could incorporate these into a more complex model. Likewise, while we have modeled the functional cycle using the deterministic replicator dynamics, language change itself is undoubtedly a stochastic process.  We could allow for the stochastic effects of finite populations \citep{moran1958,kimura1968} in the functional cycle and language  change more broadly. In both cases, we could add additional empirically-motivated parameters to the model. But, for each additional set of parameters the added complexity would have to be justified by an improvement in the model fit. 

The important characteristic of all of these models is that they can be tested by further means. For example, iterated ``generations" of cohorts playing signaling games under conflicting interests in an experimental setting should offer a means of assessing the parameters of this model of the functional cycle  (cf. \citealt{blume-etal:2001,mesoudi2008multiple}).  We could also compare the full model to altogether different models. However, it is important to emphasize the empirical content of the model we have proposed insofar as the parameters can be interpreted and assessed by further means.  Indeed, our second main point is that \emph{interpretability} should be considered alongside model fit and complexity in the comparison of models. As a case in point, we could compare the dynamic model to the loess curves in Figure \ref{func-plot} which are generated by a local regression model. At each point along the curves a polynomial function of time is fit to a local subset of the data, where data closer to the point are weighted more heavily in the fit. The parameters of the model are the relative size of the subsets and the degree of the polynomial function used. 

Now, the $AIC$ of this local regression may be much lower than that of the dynamic model, but it would be strange to posit the local regression as a model of the functional cycle. There is no sense in which the subset size, the degree of the polynomial function, or the coefficients of a series of polynomials of time can be understood in terms of the interaction between speakers and hearers or explain \emph{why} those interactions lead to a particular change.  Since it lacks a mechanism for change, the local regression model can describe but not explain the functional cycle. Even widely-used models of change such as the logistic model lack a mechanism for change \citep[4]{kroch1989}. We might assume the logistic model and, by analogy with the biological case, some kind of \emph{selective advantage} for the incoming form, but how this advantage is to be interpreted is not always clear.  In contrast, one of the greatest potential contributions of dynamic evolutionary game theoretic models to the understanding of language change is that they are constructed using explicitly causal mechanisms. Our understanding of language change depends on the application of interpretable dynamic models to time series data. Crucially, these models must be explicitly formulated to allow us to actually fit them to data and compare them to each other. Collecting more time series of the same phenomena, and more phenomena will play a crucial role in these comparisons.


\bibliography{ahern.bib}


\begin{addresses}
  \begin{address}
    Christopher Ahern \\
    University of Pennsylvania\\
    Department of Linguistics\\
    619 Williams Hall \\
    Philadelphia, PA \\
    \email{cahern@ling.upenn.edu}
  \end{address}
  \begin{address}
    Robin Clark \\
    University of Pennsylvania\\
    Department of Linguistics\\
    619 Williams Hall \\
    Philadelphia, PA\\
    \email{rclark@ling.upenn.edu}
  \end{address}
\end{addresses}


\end{document}
